\documentclass{article}
\usepackage[utf8]{inputenc}
\usepackage{array}
\usepackage{geometry}
\usepackage{stix}
\usepackage{microtype}
\usepackage[export]{adjustbox}
\usepackage{float}

%%%%%%%%%%%%%%%%%%
%%  SETTINGS    %%
%%%%%%%%%%%%%%%%%%

\geometry{%
  left=2.5cm,
  right=2.5cm,
  top=1.5cm,
  bottom=2.0cm
}
\setlength{\parindent}{0pt}
\pagestyle{empty}

\title{Dataset v0.5.0}

\begin{document}

\maketitle

\section{Description}
Each dataset are built with R1, R2 and interleaved file.\par
Bases quality are built randomly using 3 ranges : good (25-30), medium (18-21) and bad (1-3).

\section{GoldInput}
\subsection{Big - BIG}
Several groups of sequence with differents range of length :
\begin{itemize}
    \item 1500 sequences of 30 pb 
    \item 2000 sequences of 40 pb
    \item 3000 sequences of 50 pb
    \item 1500 sequences of 75 pb
    \item 2000 sequences of 100 pb
\end{itemize}
A pair between R1 and R2 are the same length.\par
Groups of length are chooseen randomly.\par
Good quality is used.\par

\subsection{Length Minimum - LENMIN}
10 sequences following :
\begin{table}[H]
\begin{tabular}{|l|c|c|}\hline
\textbf{Record} & \textbf{R1} (pb) & \textbf{R2} (pb) \\ \hline
    1 & 20 & 19 \\ \hline
    2 & 20 & 20 \\ \hline
    3 & 20 & 21 \\ \hline
    4 & 49 & 50 \\ \hline
    5 & 51 & 50 \\ \hline
    6 & 50 & 50 \\ \hline
    7 & 100 & 99 \\ \hline
    8 & 100 & 100 \\ \hline
    9 & 101 & 101 \\ \hline
\end{tabular}
\end{table}
A good quality is used.

\subsection{Quality Sliding Window - QUALSLD}
10 sequences following :
\begin{table}[H]
\begin{tabular}{|l|c|c|c|c|c|c|c|c|} \hline
    \textbf{Record} & \multicolumn{4}{c|}{R1} & \multicolumn{4}{c|}{R2} \\ \hline
    Quality & Good & Medium & Bad & Total length & Good & Medium & Bad & Total length\\ \hline
    1 & 0-5 & & & 5 & 0-15 & & & 15 \\ \hline
    2 & 0-15 & & & 15 & 0-5 & & & 5 \\ \hline
    3 & 0-74,81-100 & 75-80 & & 100 & 0-100 & & & 100 \\ \hline
    4 & 0-100 & & & 100 & 0-74,81-100 & 75-80 & & 100 \\ \hline
    5 & 0-74,81-100 & 75-80 & & 100 & 0-74,81-100 & 75-80 & & 100 \\ \hline
    6 & 0-49,71-100 & 50-70 & & 100 & 0-100 & & & 100 \\ \hline
    7 & 0-100 & & & 100 & 0-69,81-100 & 70-80 & & 100 \\ \hline
    8 & & 0-2,7-100 & 3-6 & 100 & 0-100 & & & 100 \\ \hline
\end{tabular}
\end{table}

\subsection{Quality Tail - QUALTAIL}
7 sequences following:
\begin{table}[H]
\begin{tabular}{|l|c|c|c|c|c|c|c|c|} \hline
    \textbf{Record} & \multicolumn{4}{c|}{R1} & \multicolumn{4}{c|}{R2} \\ \hline
    Quality & Good & Medium & Bad (Q30) & Total length & Good & Medium & Bad (Q30) & Total length\\ \hline
    1 & 0-5 & & & 5 & 0-15 & & & 15 \\ \hline
    2 & 0-99 & & 5 & 15 & 0-5 & & & 5 \\ \hline
    3 & 0-99 & & 100- (5) & 150 & 0-150 & & & 150 \\ \hline
    4 & 0-150 & & & 150 & 0-99 & & 100- (5) & 150 \\ \hline
    5 & 0-99 & & 100- (5) & 150 & 0-99 & & 100- (5) & 150 \\ \hline
    6 & 0-119 & & 120- (10) & 150 & 0-150 & & & 150 \\ \hline
    7 & & 0-69,81-100 & 70-80 & 150 & & 0-69,81-100 & 70-80 & 150 \\ \hline
\end{tabular}
\end{table}


\subsection{Information Dust - INFODUST}
6 sequences following:
\begin{table}[H]
\begin{tabular}{|l|c|c|c|c|} \hline
    \textbf{Record} & \multicolumn{2}{c|}{R1} & \multicolumn{2}{c|}{R2} \\ \hline
    Quality & Score & Length & Score & Length\\ \hline
    1 & 1.06 & 50 & 1.06 & 50 \\ \hline
    2 & 1.06 & 50 & 1.42 & 50 \\ \hline
    3 & 1.98 & 150 & 2.40 & 150 \\ \hline
    4 & 2.97 & 200 & 4.45 & 300 \\ \hline
    5 & 6.03 & 300 & 2.40 & 150 \\ \hline
    6 & 3.97 & 200 & 4.45 & 300 \\ \hline
\end{tabular}
\end{table}
A good quality is used.

\subsection{Information N - INFON}
4 sequences following:
\begin{table}[H]
\begin{tabular}{|l|c|c|c|c|} \hline
    \textbf{Record} & \multicolumn{2}{c|}{R1} & \multicolumn{2}{c|}{R2} \\ \hline
    Quality & N & Length & Score & Length\\ \hline
    1 & 0 & 50 & 0 & 50 \\ \hline
    2 & 1 & 50 & 0 & 150 \\ \hline
    3 & 0 & 150 & 0 & 150 \\ \hline
    4 & 2 & 200 & 3 & 300 \\ \hline
    5 & 4 & 200 & 4 & 300 \\ \hline
    6 & 0 & 300 & 0 & 300 \\ \hline
\end{tabular}
\end{table}


\section{GoldOutput}


\subsection{BIG}
\textbf{BIG-A} : keeping length from 35 pb\\
\textbf{BIG-B} : keeping length from 55 pb

\subsection{LENGTHMIN}
\textbf{LENGTHMIN-A} : remove sequences below 20] pb\\
\textbf{LENGTHMIN-B} : remove sequences below 50] pb\\
\textbf{LENGTHMIN-C} : remove sequences below 100] pb

\subsection{QUALSLD}
\textbf{QUALSLD-A} : R1+R2 : 1, 2, 7, 8 removed\\
\textbf{QUALSLD-B} : R1+R2 : 1, 2 removed, R2 : 7 truncated\\
\textbf{QUALSLD-C} : R1+R2 : 1, 2, 8 removed, R2 : 7 truncated\\
\textbf{QUALSLD-D} : R1+R2 : 1, 2, 8 removed, R2 : 7 truncated\\
\textbf{QUALSLD-E} : R1 : 8 truncated, R2 : 7 truncated\\
\textbf{QUALSLD-F} : R2 : 7 truncated

\subsection{QUALTAIL}
\textbf{QUALTAIL-A} : 6 truncated\\
\textbf{QUALTAIL-B} : 3, 7 truncated, R1 : 4 truncated, R2 : 5 truncated\\
\textbf{QUALTAIL-C} : without 7, 3 truncated, R1 : 4 truncated, R2 : 5 truncated\\
\textbf{QUALTAIL-D} : without 3, 4, 5, 7\\
\textbf{QUALTAIL-E} : without 1,2 ; 8 truncated\\
\textbf{QUALTAIL-F} : without 1, 2, 8\\

\subsection{INFODUST}
\textbf{INFODUST-A} : under 3, without 3, 4, 5\\
\textbf{INFODUST-B} : under 5, without 4
    
\subsection{INFON}
\textbf{INFON-A} : without 2, 4, 5\\
\textbf{INFON-B} : without 5\\

\subsection{Other}
\textbf{EMPTY.fastq} : empty file

\section{Tests}
\subsection{Description}
Two main groups : General, Trimmer divide in several categories\\
Categories : GenTrim, GenDiscard, GenFormat, GenThread, GenCompress, TrimLengthMin, TrimQualTail, TrimQualSld, TrimInfoDust, TrimInfoEntropy\par

For tests belong to General group, no rules exist, better combination are employed.\\
For test belong to Trimmer group, only trimming paired is test, with 4 datasets : 1 without discarded sequences and 3 removing sequences.
For each test, it's given its goal, its main and input used and output file to compare.

\subsection{GenTrim}
Goal : test if trimming works properly for trimming R1 single and R1-R2 paired.\par
Gold Input : LENGTHMIN\\
Sequencing : paired input/output
Single sequencing
\begin{table}[H]
\begin{tabular}{|l|c|c|c|c|} \hline
    \textbf{Name} & \textbf{Test} & \textbf{Trimmer} & \textbf{Output} \\ \hline
    \textbf{A} & Absence of trimming & LenMin - 10 & Same input\\ \hline
    \textbf{B} & 3 sequences trimmed & LenMin - 20 & LENGTH-A\\ \hline
    \textbf{C} & 5 sequences trimmed & LenMin - 50 & LENGTH-B\\ \hline
\end{tabular}
\end{table}
Paired sequencing
\begin{table}[H]
\begin{tabular}{|l|c|c|c|c|} \hline
    \textbf{Name} & \textbf{Test} & \textbf{Trimmer} & \textbf{Output}\\ \hline
    \textbf{D} & Absence of trimming & LenMin - 10 & Same input\\ \hline
    \textbf{E} & R1+R2 : 2, R1 : 1 & LenMin - 20 &  LENGTH-A\\ \hline
    \textbf{F} & R1+R2 : 3, R1 : 1, R2, 1 & LenMin - 50 & LENGTH-B\\ \hline
\end{tabular}
\end{table}

\subsection{GenDiscard}
Goal : check unkeep sequences.\\
Gold Input : LENGTHMIN\\
Sequencing : paired input/output
\begin{table}[H]
\begin{tabular}{|l|c|c|c|c|} \hline
    \textbf{Name} & \textbf{Test} & \textbf{Trimmer} & \textbf{Output} \\ \hline
    \textbf{A} & Absence of trimming & LenMin - 10 & Empty file \\ \hline
    \textbf{B} & R1+R2 : 2, R1 : 1 & LenMin - 20 & LENGTHMIN-A.R1.single.discard.fastq \\ \hline
    \textbf{C} & R1+R2 : 3, R1 : 1, R2, 1 & LenMin - 50 & LENGTHMIN-B.discard.fastq \\ \hline
\end{tabular}
\end{table}

\subsection{GenFormat}
Goal : test input or output are paired or interleaved matter make no differences.\\
Gold Input : LENGTHMIN
\begin{table}[H]
\begin{tabular}{|l|c|c|c|c|} \hline
    \textbf{Name} & \textbf{Test} & \textbf{Trimmer} & \textbf{Input} & \textbf{Output} \\ \hline
    \textbf{A} & Absence of trimming & LenMin - 10 & Input - Paired & Input - Interleaved \\ \hline
    \textbf{B} & Absence of trimming & LenMin - 10 & Input - Interleaved & Input - Paired \\ \hline
    \textbf{C} & R1+R2 : 3, R1 : 1, R2, 1 & LenMin - 50 & Input - Paired & LENGTHMIN-B - Interleaved\\ \hline
    \textbf{D} & R1+R2 : 3, R1 : 1, R2, 1 & LenMin - 50 & Input - Interleaved & LENGTHMIN-B - Paired\\ \hline
\end{tabular}
\end{table}

\subsection{GenThread}
Goal : test if threading modify data trimmed or order of sequences.
Gold Input : BIG\\
Sequencing : paired input/output
\begin{table}[H]
\begin{tabular}{|l|c|c|c|c|c|c|} \hline
    \textbf{Name} & \textbf{Test} & \textbf{Trimmer} & \textbf{Read batch} & \textbf{Threads} & \textbf{Output} \\ \hline
    \textbf{A} & Absence of trimming & LenMin - 10 & 100 & 2 & Same input \\ \hline
    \textbf{B} & Absence of trimming & LenMin - 10 & 100 & 4 & Same input\\ \hline
    \textbf{C} & Absence of trimming & LenMin - 10 & 100 & 6 & Same input\\ \hline
    \textbf{D} & Absence of trimming & LenMin - 10 & 100 & 8 & Same input\\ \hline

    \textbf{E} & 6500 sequences trimmed & LenMin - 55 & 100 & 2 & BIG-B\\ \hline
    \textbf{F} & 6500 sequences trimmed & LenMin - 55 & 100 & 4 & BIG-B\\ \hline
    \textbf{G} & 6500 sequences trimmed & LenMin - 55 & 100 & 6 & BIG-B\\ \hline
    \textbf{H} & 6500 sequences trimmed & LenMin - 55 & 100 & 8 & BIG-B \\ \hline
\end{tabular}
\end{table}

\subsection{GenCompress}
Goal : test compression gzip format.\\
Gold Input : BIG\\
Sequencing : paired input/output
\begin{table}[H]
\begin{tabular}{|l|c|c|c|c|} \hline
    \textbf{Name} & \textbf{Test} & \textbf{Trimmer} & \textbf{Input} & \textbf{Output}\\ \hline
    \textbf{A} & Absence of trimming & LenMin - 10 & Paired - raw & Paired - gzip\\ \hline
    \textbf{B} & Absence of trimming & LenMin - 10 & Paired - gzip & Paired - raw\\ \hline
    \textbf{C} & 6500 sequences trimmed & LenMin - 55 & Paired - raw & BIG-B - gzip\\ \hline
\end{tabular}
\end{table}

\subsection{TrimLengthMin}
Goal : test trimmer length minimum.\\
Gold Input : LENGTHMIN\\
Sequencing : paired input/output
\begin{table}[H]
\begin{tabular}{|l|c|c|c|c|} \hline
    \textbf{Name} & \textbf{Test} & \textbf{Trimmer} & \textbf{Output} \\ \hline
    \textbf{A} & Absence of trimming & 10 & Same input \\ \hline
    \textbf{B} & 5 sequences trimmed, 3 from paired, 1 from R1, 1 from R2 & 50 & LENGTH-B\\ \hline
    \textbf{C} & 8 sequences trimmed & 100 & LENGTH-C\\ \hline
\end{tabular}
\end{table}

\subsection{TrimQualSld}
Goal : test trimmer quality sliding window on end of read\\
Gold Input : QUALSLD\\
Sequencing : paired input/output
\begin{table}[H]
\begin{tabular}{|l|c|c|c|c|} \hline
    \textbf{Name} & \textbf{Test} & \textbf{Trimmer} & \textbf{Output} \\ \hline
    \textbf{A} & Absence of trimming & 1:5 & Same input \\ \hline
    \textbf{B} & Remove on R1 and R2 & 20:25 & QUALSLD-A\\ \hline
    \textbf{C} & Remove on both, truncated on 1 side & 10:10 & QUALSLD-B\\ \hline
    \textbf{D} & Remove on both, truncated on 1 side & 17:4 & QUALSLD-C\\ \hline
    \textbf{E} & Remove on both, truncated on 1 side & 17:10 & QUALSLD-D\\ \hline
    \textbf{F} & Truncated on both side & 4:1 & QUALSLD-E\\ \hline
    \textbf{G} & Truncated only on 1 side & 4:4 & QUALSLD-F\\ \hline
\end{tabular}
\end{table}

\subsection{TrimQualTail}
Goal : test trimmer quality on tail.\\
Gold Input : QUALTAIL\\
Sequencing : paired input/output
\begin{table}[H]
\begin{tabular}{|l|c|c|c|c|} \hline
    \textbf{Name} & \textbf{Test} & \textbf{Trimmer} & \textbf{Output} \\ \hline
    \textbf{A} & Absence of trimming & 2:2 & Same input \\ \hline
    \textbf{B} & 3 sequences truncated & 5:2 & QUALTAIL-A\\ \hline
    \textbf{C} & 3 sequences truncated & 5:2:60 & QUALTAIL-A\\ \hline
    \textbf{D} & 3 sequences truncated & 5:5:60 & QUALTAIL-A\\ \hline
    \textbf{E} & 3 sequences trimmed & 5:5:70 & QUALTAIL-B\\ \hline
    \textbf{F} & 1 sequence truncated & 3:10:79 & QUALTAIL-C\\ \hline
    \textbf{G} & 1 sequence trimmed & 3:10:81 & QUALTAIL-C\\ \hline
\end{tabular}
\end{table}

\subsection{TrimInfoDust}
Goal : test trimmer information dust.\\
Gold Input : INFODUST\\
Sequencing : paired input/output
\begin{table}[H]
\begin{tabular}{|l|c|c|c|c|} \hline
    \textbf{Name} & \textbf{Test} & \textbf{Trimmer} & \textbf{Output} \\ \hline
    \textbf{A} & Absence of trimming & 1 & Same input\\ \hline
    \textbf{B} & 3 sequences trimmed (1,2,3) & 2 & INFODUST-A\\ \hline
    \textbf{C} & 5 sequences trimmed (1,2,3,4,5) & 3 & INFODUST-B\\ \hline
\end{tabular}
\end{table}

\subsection{TrimInfoN}
Goal : test trimmer information N.\\
Gold Input : INFON\\
Sequencing : paired input/output
\begin{table}[H]
\begin{tabular}{|l|c|c|c|c|} \hline
    \textbf{Name} & \textbf{Test} & \textbf{Trimmer} & \textbf{Output} \\ \hline
    \textbf{A} & Absence of trimming & 5 & Same input\\ \hline
    \textbf{B} & 3 sequences trimmed & 1 & INFON-A\\ \hline
    \textbf{C} & 1 sequence trimmed & 4 & INFON-B\\ \hline
\end{tabular}
\end{table}

\end{document}
